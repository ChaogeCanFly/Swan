
\section{Session 3: The SLERP Method}

\begin{center}
Albert Torres Rubio \\
11/10/2017
\end{center}

The SLERP algorithm is used in Topology Optimization when a level-set function describes the topology of the domain. It consist in using the Topological Derivative to minimize a function ($J(\chi)$) in the whole domain providing a topology which has optimal stiffness for a certain value of volume. In the next paragraphs this method is explained.

Firstly, the Topological Derivative has to be defined as the difference between the actual state of the topology and any infinitesimal change in it. Equation (\ref{E:TD}) represents the definition of this derivative.

\begin{equation}  \label{E:TD}
J\left(\chi + \tilde{\chi}_\epsilon\right)-J(\chi)=D_TJ(\chi) \: |Be| 
\end{equation}

Where,
\begin{itemize}
\item $\tilde{\chi}_\epsilon$ is any infinitesimal topology change.
\item $J(\chi)$ is the actual topology.
\item $D_T$ is the Topological Derivative.
\item $Be$ is the ball inserted in the domain
\end{itemize}

The aim of the whole process is to minimize the topology function, therefore, the optimal condition is expressed as Equation \ref{E:Optimal} states.

\begin{equation}  \label{E:Optimal}
J\left(\chi + \tilde{\chi}_\epsilon\right)-J(\chi)=D_TJ(\chi) \: |Be| \geq 0  \qquad \forall \:\tilde{\chi}
\end{equation}

Following reference work a $g$ function is defined from the Topological Derivative as:

\begin{equation}  \label{E:g}
g=\left\{ 
\begin{array}{c}
\quad D_TJ \qquad \Psi > 0 \quad (X \in \Omega^-) \\
-D_TJ \qquad \Psi < 0 \quad (X \in \Omega^+) \\
\end{array} \right.
\end{equation}
Notice that $\Psi$ is the level-set function that defines the topology. As a result, the new definition of the optimal condition becomes:

\begin{equation}  
g(\chi)\left\{ 
\begin{array}{c}
>0 \qquad \Psi > 0 \quad (\chi \in \Omega^-) \\
<0 \qquad \Psi < 0 \quad (\chi \in \Omega^+) \\
\end{array} \right.
\end{equation}

Alternative as:

\begin{equation}  \label{E:debil}
sign(g(\chi(\Psi)))=sign(\Psi)
\end{equation}

Thus, in order to find the optimal solution only the signs must coincide between g and $\Psi$. Moreover, in order to solve the problem, a strong optimality condition is imposed through the following requirements.

\begin{enumerate}
\item To satisfy the same sign of the gradient and the level-set, it is imposed to be the same.
\item Transform the optimal condition represented in Equation (\ref{E:debil}) as the following expression:
\begin{equation}  \label{E:forta}
R(\Psi)=g(\chi(\Psi^*))-\Psi^*=0
\end{equation}
By this new definition, the problem can be solved by a Fix Point method, performing the iteration: $\Psi_{n+1}=g(\chi(\Psi_n))$.

\item  Reduce level-set possible infinite values by making unitary its $L^2(\Omega)$ norm:
\begin{equation}  \label{E:modul}
||\psi||_L^2(\Omega)=1 \quad \rightarrow  \quad (\Psi,\Psi )=\int_{\Omega} \Psi^2 = 1
\end{equation}
Where $(\cdot,\cdot)$ is the scalar product.
\end{enumerate}

Having introduced these changes, we can rewrite Equation (\ref{E:forta}) in its general case as:
\begin{equation}  \label{E:Rnew}
R(\Psi)=\frac{g(\chi(\Psi))}{||g(\chi(\Psi))||}-\Psi=0
\end{equation}

It must be commented that when the optimal solution is reached using Equation (\ref{E:Rnew}), $||g(\chi(\Psi))||$ becomes equal to 1. Therefore, Equation (\ref{E:forta}), which is the original optimal condition remains invariant. Trying to solve the problem with this configuration with a Fix-point method would lead to:

\begin{equation}  \label{E:iteration}
\Psi_{n+1}=\frac{g(\chi(\Psi_n))}{||g(\chi(\Psi_n))||}
\end{equation}

The above equation would be too aggressive to solve directly in some cases. Thus, a relaxation the algorithm (\ref{E:iteration}) is proposed through a line-search method. The values $\alpha$ and $\beta$ are computed by explicitly impose $||\psi||=1$, leading to the final expression of the SLERP algorithm (\ref{E:alpha}).


\begin{equation}  \label{E:alpha}
\Psi_{n+1}=\beta\Psi_n+\alpha\frac{g(\chi(\Psi_n))}{||g(\chi(\Psi_n))||}
\end{equation}

Where $ \theta_n = \arccos \left(\frac{(\dot{\Psi}_n,g_n)}{||\Psi_n||||g_n||}\right)$ and plays the role of the line search parameter.

%Furthermore, Equation \ref{E:alpha} is then now solved but not using its expression directly. To do so, it must be said that $\Psi_{n}$ and $g(\chi(\Psi_n))$ can be represented as the legs of a triangle which can be added as vectors, resulting $\Psi_{n+1}$ the solution over a plane. In addition, a $\kappa$ parameter is introduced which includes $\alpha$ and $\beta$ values. This parameter, as a angle measure,  assigns the portion of $\Psi_{n}$ and $g(\chi(\Psi_n))$ which take part in the next solution of the iterative process. The next solution is found by means of the law of sines and eventually, the optimal solution is found by making the global residue 0.

\clearpage
















